\documentclass[a4paper,11pt]{article}

\usepackage[utf8]{inputenc}

\usepackage{minted}

\usepackage{hyperref}

\usepackage{float}

\usepackage{tikz}

\usepackage{amsmath}

\usepackage{pgfplots}
\pgfplotsset{compat=newest, scaled y ticks=false}

\begin{document}

\title{
    \textbf{Doubly linked lists}
}
\author{Hannes Mann}
\date{2022-09-30}

\maketitle

\section*{Introduction}

This is a report for the first higher grade assignment in the course ID1021 Algorithms and Data Structures.
In this assignment, the benefits and drawbacks of doubly linked lists compared to singly linked lists should be explored.

The implementation was done in C++ and was compiled with GCC using the {\tt -O0} parameter to ensure the compiler doesn't optimize away any part of the code.

The source code can be found at: \href{https://github.com/HannesMann/ID1021/tree/main/src/doubly_linked_list}{GitHub}.

\section*{Implementing a doubly linked list}

A doubly linked list is very similar to a ''regular'' singly linked list but with an important difference: Each element has a pointer to both the previous and next element.
This means that the list can be traversed in both directions.
An element with no ''previous'' pointer is the first element in the list, and an element with no ''next'' pointer is the last element in the list.

\begin{minted}{cpp}
template<typename T>
struct DoublyLinkedList {
	DoublyLinkedList(T item) { ... }

	T item;
	DoublyLinkedList<T> *previous, *next;

	DoublyLinkedList<T>* first() {
		/* While previous element exists, continue... */
	}
	DoublyLinkedList<T>* last() {
		/* While next element exists, continue... */
	}

	/* Append element to end of list */
	void append_to_end(DoublyLinkedList<T>* list) {
		/* Update last()->next and item->previous... */
	}

	/* Remove this item from the list */
	void remove_self() {
		/* Update previous->next and next->previous... */
	}
}
\end{minted}

\section*{Comparison to singly linked lists}

When comparing doubly linked lists to singly linked lists, the first thing that is immediately obvious is that doubly linked lists will use more memory.
Singly linked lists store one item of type $T$ and a next pointer, while a doubly linked list needs to store an additional pointer for every element.
This pointer requires an extra 4 or 8 bytes on modern architectures.

The additional information that an element in a doubly linked list carries does however allow some operations to be more efficient.
Removing an element in a singly linked list means that the first element in the list needs to be known, and from this first element the list is searched for the element to remove.

With a doubly linked list an item has the ability to remove itself from the list without needing to search through it.
The neighbors are updated so that the ''previous'' element points to the ''next'' element, and vice versa.
This brings a remove operation down from $O(n)$ to $O(1)$ which can be a massive benefit in large lists.

Adding an element to a doubly linked list requires an additional pointer update but this is trivial on modern hardware. The operation remains $O(1)$ but the execution time doubles.

\section*{Benchmarks}

Benchmarking is done by measuring the time it takes to remove and add back a random element from a list with $n$ elements.
The random element is picked in constant time and the sequence is the same for every run of the benchmark.

\begin{table}[H]
\centering
\begin{tabular}{|c|c|c|}
\hline
\textbf{n} & \textbf{singly linked} & \textbf{doubly linked} \\
\hline
	5 & 5.5 ns & 7.1 ns \\
	50 & 34 ns & 7.6 ns \\
	100 & 67 ns & 7.2 ns \\
	1000 & 1200 ns & 7.2 ns \\
	10000 & 16000 ns & 8.2 ns \\
	50000 & 60000 ns & 9.0 ns \\
	100000 & 120000 ns & 8.8 ns \\
	250000 & 280000 ns & 14 ns \\
\hline
\end{tabular}
\caption{Execution time for removing and adding elements to list of size $n$.}
\label{tab:table1}
\end{table}

\begin{table}[H]
\centering
\begin{tikzpicture}
\begin{axis}
[
	xlabel={n},
	ylabel={t},
	grid,
	xmin=5,
	xmax=250000,
	ymin=5.5,
	ymax=280000,
	y label style={rotate=-90}
]
\addplot[red, no marks, densely dashed] table[col sep=comma]{plot1.csv};
\addlegendentry{singly linked}
\addplot[blue, no marks, densely dashed] table[col sep=comma]{plot2.csv};
\addlegendentry{doubly linked}
\end{axis}
\end{tikzpicture}
\end{table}

As can be seen from the results removing elements from a doubly linked list is much faster than the same operation on a singly linked list.
The operation executes in linear time for the singly linked list and constant time for the doubly linked list.

\section*{Observations from the benchmark}

The first thing to note is that for very small values of $n$ the singly linked list is actually faster. This is because there are less pointers to update.

It's also interesting to note that the doubly linked list sees a small time increase when $n=250000$, even though the same amount of operations are executed.

This likely has to do with the benchmark itself and not the linked list implementation.
Because the benchmark uses an array to select a random element in constant time, what's happening is likely that the array no longer fits in L1/L2/etc.\ cache like it did when $n$ was smaller and because of this accessing the ''random index'' array is what is responsible for the time increase.

\section*{Conclusion}

Doubly linked lists have obvious advantages when list manipulation or searching is required and the extra memory required doesn't matter much unless the system is heavily memory constrained.
For most use cases, doubly linked lists seem like the better choice.

\end{document}